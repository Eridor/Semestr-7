\documentclass[12pt,a4paper,notitlepage]{report}
\usepackage{polski}
\usepackage[utf8]{inputenc}
\usepackage[polish]{babel}
\usepackage{url}
\usepackage[pdftex]{graphicx}
\usepackage{sidecap}
\usepackage{verbatim}
\usepackage{array}
\usepackage{amsmath}
\usepackage{enumitem}
\usepackage{fancyhdr}
\usepackage[margin=1.8cm]{geometry}
\usepackage{float}
\usepackage{listings}
\usepackage{color}
\usepackage{wrapfig,booktabs}
\usepackage{caption}
\usepackage{wrapfig}

\newcommand\blfootnote[1]{%
  \begingroup
  \renewcommand\thefootnote{}\footnote{#1}%
  \addtocounter{footnote}{-1}%
  \endgroup
}
%remove section whitespace
\usepackage{titlesec}
\titlespacing\section{0pt}{12pt plus 4pt minus 2pt}{0pt plus 2pt minus 2pt}

\makeatletter
\newcommand*{\toccontents}{\@starttoc{toc}}
\makeatother

\renewcommand\thesection{\arabic{section}}
\newcommand{\HRule}{\rule{\linewidth}{0.5mm}} %Do tytułu

\usepackage{etoolbox} %do bibliografi
\patchcmd{\thebibliography}{\chapter*}{\section*}{}{} %żeby bibliografię nie wrzucało na nową stronę


%Do kodu C++
\usepackage{xcolor} % for setting colors

% set the default code style
\lstset{
    frame=tb, % draw a frame at the top and bottom of the code block
    tabsize=4, % tab space width
    showstringspaces=false, % don't mark spaces in strings
    numbers=left, % display line numbers on the left
    breaklines=true,                % sets automatic line breaking
    commentstyle=\color{green}, % comment color
    keywordstyle=\color{blue}, % keyword color
    stringstyle=\color{red} % string color
}
%End kod

\usepackage{amsmath} % Required for \varPsi below
\usepackage{tikz}
\usepackage{varwidth}
\usetikzlibrary{shapes,arrows}
\usepackage{pgfplots}

\usepackage{xfrac} %do ułamków np.: 1\2

\begin{document}
\renewcommand{\lstlistingname}{Kod} %Zmiana listing na Kod
\renewcommand\bibname{Literatura} %Zmiana "Bibliografia" na "Literatura"
\renewcommand{\tablename}{Tabela} %Zmiana "Tablica" na "Tabela"
%\renewcommand{\figurename}{Zrzut ekranu} %Zmiana figure na Zrzut ekranu

\begin{titlepage}
\begin{center}

\textsc{\LARGE Politechnika Poznańska}\\[0.8cm]
\textsc{\Large Wydział Elektryczny \\ Informatyka}\\[1cm]

\includegraphics[scale=1]{logopoli}
\vspace{.5cm}\\
\textsc{\large Teoria Informacji i Kodowanie}\\[0.2cm]
\textsc{\normalsize Dokumentacja Projektu}\\[.5cm]


% Author and supervisor
\noindent
\begin{minipage}[t]{0.4\textwidth}
\begin{flushleft} \large
\emph{Autorzy:}\\
Michał \textsc{Majka} \\
Nr albumu: 112679\\
Piotr \textsc{Parysek} \\
Nr albumu: 106100
\end{flushleft}
\end{minipage}%
\begin{minipage}[t]{0.4\textwidth}
\begin{flushright} \large
\emph{Prowadzący:} \\
dr inż. Ewa \textsc{Idzikowska}
\end{flushright}
\end{minipage}

%\bigskip
\vspace{2cm}
% Bottom of the page
{\large \today}
%\global\let\newpagegood\newpage %żeby nie było page break po titilepage
%\global\let\newpage\relax %żeby nie było page break po titilepage

\end{center}
\end{titlepage}

\toccontents

\section{Wstęp}
Zadaniem projektowym była implementacja algorytmu kompresji bezstratnej \newline \emph{\textbf{Run-Length Encoding}} (\textbf{RLE}).

Zadanie zrealizowaliśmy w środowisku programistycznym Qt Creator 5.5.1\cite{qt} korzystając z kompilatora GCC 4.9.1\cite{gcc}.\\
Do kontroli wersji oraz plików źródłowych wykorzystaliśmy oprogramowanie Git\cite{git}, projekt hostowaliśmy w repozytorium GitHub\cite{github}.\\
Dokumentacja została wykonana w \LaTeX \cite{latex} w programie Texmaker 4.5 \cite{program} oraz edytora online: ShareLaTeX\cite{sharelatex}. 

\section{Algorytm}
\subsection{Historia}
\textbf{Run-Length Encodings}, również znane jako \textbf{Golomb Codings}, swoje ,,podwaliny'' powstania wiąże z pracami XVII-wiecznego francuskiego matematyka \textit{Blaise'a Pascal'a} związanych z probabilistyką\cite{book}. Koncepcja kodowania powtarzających się znaków była używana od początków istnienia teorii informacji (Shannon 1949, Laemmel 1951), jednakże metodę oraz sposób kodowania wynalazł i opracował \textit{Solomon Wolf Golomb}\cite{golomb}.

\subsection{Zasada działania}
Algorytm jest relatywnie prosty $\rightarrow$ przedstawia powtarzające się wartości jako dany znak i licznik powtórzeń. Na przykład ciąg znaków:
\[pppppppuuuuuttttt......ppppoooooozzzzznnnnannnnn....pppppplllll\]
Zostaje przedstawiony w ciąg postaci:
\[p7u5t5.6p4o6z5n4an5.4p6l5\]

\begin{table}[H]
\begin{tabular}{l l}
Ciąg znaków & Liczba \\\hline
$pppppppuuuuuttttt......ppppoooooozzzzznnnnannnnn....pppppplllll$& 64\\
$p7u5t5.6p4o6z5n4an5.4p6l5$& 26 \\ \hline
& $\sfrac{26}{64}\approx 0.41$
\end{tabular}
\caption{Przykładowa kompresja znakowa}
\end{table}

\section{Opis implementacji}
\subsection{Kodowanie}
% Define block styles
\tikzstyle{decision} = [diamond, draw, fill=green!20, text width=4.5em, text badly centered, node distance=3cm, inner sep=0pt]
\tikzstyle{block} = [rectangle, draw, fill=blue!20, text centered, rounded corners, minimum height=2.5em, execute at begin node={\begin{varwidth}{15em}}, execute at end node={\end{varwidth}}]
\tikzstyle{line} = [draw, -latex']
\subsubsection{Ustalenie znaku kodowania}    
\begin{figure}[H]
\centering
\begin{tikzpicture}[node distance = 2cm, auto]
    % Place nodes
    \node [block] (init) {Zapisanie do QMap wszystkich możliwych wartości bajtu [256 możliwości], wraz z liczbą wystąpień = 0};
    \node [block, below of=init, node distance=2.5cm] (identify) {Wylicznie wystąpień każdego bajtu i zmiana poszczególnych wartości w QMap};
    \node [block, below of=identify] (evaluate) {Wybierz wartość ,,znaku''};
    \node [block, left of=evaluate, node distance=5cm] (update) {Zmień ,,znak''};
    \node [decision, below of=update] (range) {Czy osiągniętko koniec zakresu?};
    \node [decision, below of=evaluate] (decide) {Czy liczba wystąpień znaku jest równa 0?};
    \node [block, below of=decide, node distance=3cm] (stop) {Zapisz znak w zmiennej SIGN};
    \node [block, below of=range, node distance=4.5cm] (qset) {Stwórz QSet wszyskich wystąpień dwubajtowych w danym pliku};
    \node [block, below of=qset] (choose) {Wybierz wartość dwu Bajtową ,,znaku''};
    \node [decision, below of=choose] (signdecide) {Czy ,,znak'' jest w QSet?};
    \node [block, below of=signdecide, node distance=3cm] (over) {Zapisz znak w zmiennej SIGN16};
    \node [block, right of=signdecide, node distance=5cm] (notyet) {Zmień wartość ,,znaku''};
    \node [block, below of=stop] (end) {Koniec.};
    \node [block, right of=over, node distance=5cm] (theend) {Koniec};
    % Draw edges
    \path [line] (init) -- (identify);
    \path [line] (identify) -- (evaluate);
    \path [line] (evaluate) -- (decide);
    \path [line] (decide) -- node [near start] {Nie} (range);
    \path [line] (range) -- node [near start] {Nie} (update);
    \path [line] (update) |- (evaluate);
    \path [line] (decide) -- node {Tak} (stop);
    \path [line] (range) -- node [near start] {Tak} (qset);
    \path [line] (qset) -- (choose);
    \path [line] (choose) -- (signdecide);
    \path [line] (signdecide) -- node [near start] {Nie} (over);
    \path [line] (signdecide) -- node [near start] {Tak} (notyet);
    \path [line] (notyet) |- (choose);
    \path [line] (stop) -- (end);
    \path [line] (over) -- (theend);
\end{tikzpicture}
\caption{Schemat blokowy wyszukiwania ,,znaku''.}
\end{figure}

\subsubsection{Kodowanie}
\begin{figure}[H]
\centering
\begin{tikzpicture}[node distance = 2cm,auto]
    % Place nodes
    \node [block] (start) {Dopisz do tablicy pliku wyjściowego rozmiar ,,znaku'' oraz Bajt/y ,,znak/u''};
    \node [block, below of=start] (jeden) {Wczytaj kolejny bajt};
    \node [decision, below of=jeden] (decyzja) {Czy bajt różni się od poprzedniego?};
    \node [block, right of=decyzja, node distance=5cm] (dwa) {Dopisz Bajt\\ do Listy Elementów};
    \node [block, left of=decyzja, node distance=5cm] (trzy) {Zinkrementuj liczbę\\ wystąpień ostatniego\\ elementu listy};
    \node [block, below of=decyzja, node distance=3.5cm] (cztery) {Zapisz Listę Elementów do Tablicy bajtów wyjściowych};
    \node [block, below of=cztery] (piec) {Zapisz tablicę do pliku};
	\node [block, below of=piec] (szesc) {Koniec.};    
    
    % Draw edges
    \path [line] (start) -- (jeden);
    \path [line] (jeden) -- (decyzja);
    \path [line] (decyzja) -- node [near start] {Tak} (dwa);
    \path [line] (decyzja) -- node [near start] {Nie} (trzy);
    \path [line] (decyzja) -- node [near start] {Koniec tablicy} (cztery);
    \path [line] (dwa) |- (jeden);
    \path [line] (trzy) |- (jeden);
    \path [line] (cztery) -- (piec);
    \path [line] (piec) -- (szesc);        
\end{tikzpicture}
\caption{Schemat blokowy kodowania pliku.}
\end{figure}

\subsection{Dekodowanie}
\begin{figure}[H]
\centering
\begin{tikzpicture}[node distance = 2cm,auto]
    % Place nodes
    \node [block] (start) {Wczytaj rozmiar ,,znaku''};
    \node [block, below of=start] (zero) {Wczytaj ,,znak''};
    \node [block, below of=zero] (jeden) {Wczytaj bajt};
    \node [decision, below of=jeden] (decyzja) {Czy bajt jest rónwy ,,znakowi''?};
    \node [block, right of=decyzja, node distance=5cm] (dwa) {Dopisz Bajt do\\ tablicy bajtów wyjściowych};
    \node [block, left of=decyzja, node distance=5cm] (trzy) {Wczytaj kolejny bajt\\ -powtarzany bajt-\\ i następny bajt -licznik\\ powtórzenia. Dopisz do\\ tablicy bajtów wyjściowych\\ powtarzany bajt w ilości\\ licznika powtórzenia.};
    \node [block, below of=decyzja, node distance=3cm] (cztery) {Zapisz tablicę do pliku};
    \node [block, below of=cztery] (piec) {Koniec.};
    
    % Draw edges
    \path [line] (start) -- (zero);
    \path [line] (zero) -- (jeden);
    \path [line] (jeden) -- (decyzja);
    \path [line] (decyzja) -- node [near start] {Nie} (dwa);
    \path [line] (decyzja) -- node [near start] {Tak} (trzy);
    \path [line] (decyzja) -- node [near start] {Koniec tablicy} (cztery);
    \path [line] (dwa) |- (jeden);
    \path [line] (trzy) |- (jeden);
    \path [line] (cztery) -- (piec);      
\end{tikzpicture}
\caption{Schemat blokowy dekodowania pliku.}
\end{figure}

\section{Użytkowanie programu}
\section{Testy}
\section{Porównanie kompresji}

\addcontentsline{toc}{section}{Literatura}  %dodanie znacznika "Literatura" do spisu treści
\begin{thebibliography}{1}
	\bibitem{latex} Kurs \LaTeX w $\pi^e$ minut \url{http://www.fuw.edu.pl/~kostecki/kurs_latexa.pdf}.
	\bibitem{program} Program Texmaker 4.5 \url{http://www.xm1math.net/texmaker/}.
	\bibitem{sharelatex} ShareLaTeX is an online LaTeX editor \url{https://www.sharelatex.com/}.	
	\bibitem{qt} Qt \url{http://www.qt.io/}.
	\bibitem{gcc} GCC, the GNU Compiler Collection \url{https://gcc.gnu.org/}.
	\bibitem{git} Git \url{http://git-scm.com/}.
	\bibitem{github} GitHub \url{https://github.com/}.
	\bibitem{golomb} Run-length encodings - S. W. Golomb (1966); IEEE Trans Info Theory 12(3):399 \url{http://urchin.earth.li/~twic/Golombs_Original_Paper/}.
	\bibitem{book} Variable-length codes for data compression / David Salomon, London : Springer, 2007.
	\bibitem{ikony} Ikony \url{http://www.flaticon.com/}.
\end{thebibliography}

	\blfootnote{Dokument wykonany w \LaTeX \cite{latex} \\ w programie Texmaker 4.5 \cite{program}}	
\end{document}
%\begin{lstlisting}[language=C++, caption={Zmieniony algorytm wielowątkowy liczenia liczby $\pi$}]
%\end{lstlisting}