\documentclass[12pt,a4paper,notitlepage]{report}
\usepackage{polski}
\usepackage[utf8]{inputenc}
\usepackage[polish]{babel}
\usepackage{url}
\usepackage[pdftex]{graphicx}
\usepackage{sidecap}
\usepackage{verbatim}
\usepackage{array}
\usepackage{amsmath}
\usepackage{enumitem}
\usepackage{fancyhdr}
\usepackage{geometry}
\newgeometry{tmargin=2.5cm, bmargin=2.5cm, lmargin=2.7cm, rmargin=2.5cm}%marginesy
\usepackage{float}
\usepackage{listings}
\usepackage{color}
\usepackage{wrapfig,booktabs}
\usepackage{caption}
\usepackage{wrapfig}

\newcommand\blfootnote[1]{%
  \begingroup
  \renewcommand\thefootnote{}\footnote{#1}%
  \addtocounter{footnote}{-1}%
  \endgroup
}
%remove section whitespace
\usepackage{titlesec}
\titlespacing\section{0pt}{12pt plus 4pt minus 2pt}{0pt plus 2pt minus 2pt}

\makeatletter
\newcommand*{\toccontents}{\@starttoc{toc}}
\makeatother

\renewcommand\thesection{\arabic{section}}
\newcommand{\HRule}{\rule{\linewidth}{0.5mm}} %Do tytułu

\usepackage{etoolbox} %do bibliografi
\patchcmd{\thebibliography}{\chapter*}{\section*}{}{} %żeby bibliografię nie wrzucało na nową stronę


%Do kodu C++
\usepackage{xcolor} % for setting colors

% set the default code style
\lstset{
    frame=tb, % draw a frame at the top and bottom of the code block
    tabsize=4, % tab space width
    showstringspaces=false, % don't mark spaces in strings
    numbers=left, % display line numbers on the left
    breaklines=true,                % sets automatic line breaking
    captionpos=b, 	%caption bottom
    commentstyle=\color{green}, % comment color
    keywordstyle=\color{blue}, % keyword color
    stringstyle=\color{red} % string color
}
%End kod

\usepackage{amsmath} % Required for \varPsi below
\usepackage{tikz}
\usepackage{varwidth}
\usetikzlibrary{shapes,arrows}
\usepackage{pgfplots}

\usepackage{xfrac} %do ułamków np.: 1\2

\usepackage{longtable} %Do tabli, mogłem się spodziewać

\usepackage{subcaption} %do kilku obrazków obok siebie

\begin{document}
\renewcommand{\lstlistingname}{Kod} %Zmiana listing na Kod
\renewcommand\bibname{Literatura} %Zmiana "Bibliografia" na "Literatura"
\renewcommand{\tablename}{Tabela} %Zmiana "Tablica" na "Tabela"
%\renewcommand{\figurename}{Zrzut ekranu} %Zmiana figure na Zrzut ekranu

\begin{titlepage}
\begin{center}

\textsc{\LARGE Politechnika Poznańska}\\[0.8cm]
\textsc{\Large Wydział Elektryczny \\ Informatyka}\\[1cm]

\includegraphics[scale=1]{logopoli}
\vspace{.5cm}\\
\textsc{\large Teoria Informacji i Kodowanie}\\[0.2cm]
\textsc{\normalsize Dokumentacja Projektu}\\[.5cm]


% Author and supervisor
\noindent
\begin{minipage}[t]{0.4\textwidth}
\begin{flushleft} \large
\emph{Autorzy:}\\
Michał \textsc{Majka} \\
Nr albumu: 112679\\
Piotr \textsc{Parysek} \\
Nr albumu: 106100
\end{flushleft}
\end{minipage}%
\begin{minipage}[t]{0.4\textwidth}
\begin{flushright} \large
\emph{Prowadzący:} \\
dr inż. Ewa \textsc{Idzikowska}
\end{flushright}
\end{minipage}

%\bigskip
\vspace{2cm}
% Bottom of the page
{\large \today}
%\global\let\newpagegood\newpage %żeby nie było page break po titilepage
%\global\let\newpage\relax %żeby nie było page break po titilepage

\end{center}
\end{titlepage}

\toccontents

\section{Wstęp}
Zadaniem projektowym była implementacja algorytmu kompresji bezstratnej \newline \emph{\textbf{Run-Length Encoding}} (\textbf{RLE}).

Zadanie zrealizowano w środowisku programistycznym Qt Creator 5.5.1\cite{qt} korzystając z kompilatora GCC 4.9.1\cite{gcc}.\\
Do kontroli wersji oraz plików źródłowych wykorzystano oprogramowanie Git\cite{git}, projekt hostowano w repozytorium GitHub\cite{github}.\\
Dokumentację wykonano w \LaTeX \cite{latex} w programie Texmaker 4.5 \cite{program} oraz w edytorze online: ShareLaTeX\cite{sharelatex}. 

\section{Algorytm}
\subsection{Historia}
\textbf{Run-Length Encodings}, również znane jako \textbf{Golomb Codings}, swoje ,,podwaliny'' powstania wiąże z pracami, XVII-wiecznego francuskiego matematyka \textit{Blaise'a Pascal'a}, związanymi z probabilistyką\cite{book}. Koncepcja kodowania powtarzających się znaków była używana od początków istnienia teorii informacji (Shannon 1949, Laemmel 1951), jednakże metodę oraz sposób kodowania wynalazł i opracował \textit{Solomon Wolf Golomb}\cite{golomb}\cite{book}.

\subsection{Zasada działania}
Algorytm jest relatywnie prosty $\rightarrow$ przedstawia powtarzające się wartości jako dany znak i licznik powtórzeń. Na przykład ciąg znaków:
\[pppppppuuuuuttttt......ppppoooooozzzzznnnnannnnn....pppppplllll\]
Zostaje przedstawiony w postaci ciągu:
\[p7u5t5.6p4o6z5n4an5.4p6l5\]

\begin{table}[H]
\begin{tabular}{l l}
Ciąg znaków & Liczba \\\hline
$pppppppuuuuuttttt......ppppoooooozzzzznnnnannnnn....pppppplllll$& 64\\
$p7u5t5.6p4o6z5n4an5.4p6l5$& 26 \\ \hline
& $\sfrac{26}{64}\approx 0.41$
\end{tabular}
\caption{Przykładowa kompresja znakowa}
\end{table}

\section{Opis implementacji}
%Program wykonano w języku C++ z metodami dostarczonymi z Qt.
\subsection{Kodowanie}
% Define block styles
\tikzstyle{decision} = [diamond, draw, fill=green!20, text width=4.5em, text badly centered, node distance=3cm, inner sep=0pt]
\tikzstyle{block} = [rectangle, draw, fill=blue!20, text centered, rounded corners, minimum height=2.5em, execute at begin node={\begin{varwidth}{15em}}, execute at end node={\end{varwidth}}]
\tikzstyle{line} = [draw, -latex']
\subsubsection{Ustalenie znaku kodowania}    
\begin{figure}[H]
\centering
\begin{tikzpicture}[node distance = 2cm, auto]
    % Place nodes
    \node [block] (init) {Zapisanie do QMap wszystkich możliwych wartości bajtu [256 możliwości], wraz z liczbą wystąpień = 0};
    \node [block, below of=init, node distance=2.5cm] (identify) {Wylicznie wystąpień każdego bajtu i zmiana poszczególnych wartości w QMap};
    \node [block, below of=identify] (evaluate) {Wybierz wartość ,,znaku''};
    \node [block, left of=evaluate, node distance=5cm] (update) {Zmień ,,znak''};
    \node [decision, below of=update] (range) {Czy osiągnięto koniec zakresu?};
    \node [decision, below of=evaluate] (decide) {Czy liczba wystąpień znaku jest równa 0?};
    \node [block, below of=decide, node distance=3cm] (stop) {Zapisz znak w zmiennej SIGN};
    \node [block, below of=range, node distance=4.5cm] (qset) {Stwórz QSet wszyskich wystąpień dwubajtowych w danym pliku};
    \node [block, below of=qset] (choose) {Wybierz wartość dwu Bajtową ,,znaku''};
    \node [decision, below of=choose] (signdecide) {Czy ,,znak'' jest w QSet?};
    \node [block, below of=signdecide, node distance=3cm] (over) {Zapisz znak w zmiennej SIGN16};
    \node [block, right of=signdecide, node distance=5cm] (notyet) {Zmień wartość ,,znaku''};
    \node [block, below of=stop] (end) {Koniec.};
    \node [block, right of=over, node distance=5cm] (theend) {Koniec};
    % Draw edges
    \path [line] (init) -- (identify);
    \path [line] (identify) -- (evaluate);
    \path [line] (evaluate) -- (decide);
    \path [line] (decide) -- node [near start] {Nie} (range);
    \path [line] (range) -- node [near start] {Nie} (update);
    \path [line] (update) |- (evaluate);
    \path [line] (decide) -- node {Tak} (stop);
    \path [line] (range) -- node [near start] {Tak} (qset);
    \path [line] (qset) -- (choose);
    \path [line] (choose) -- (signdecide);
    \path [line] (signdecide) -- node [near start] {Nie} (over);
    \path [line] (signdecide) -- node [near start] {Tak} (notyet);
    \path [line] (notyet) |- (choose);
    \path [line] (stop) -- (end);
    \path [line] (over) -- (theend);
\end{tikzpicture}
\caption{Schemat blokowy wyszukiwania ,,znaku''.}
\end{figure}
Do wyszukania ,,znaków'' wykorzystano kontener \lstinline;QMap<quint, int>;, gdzie zmienna \lstinline;quint8; (\lstinline;unsigned byte;) wskazuje na poszczególne możliwe wartości bajta, a zmienna \lstinline;int; wskazuje ilość wystąpień danego bajta w opracowywanym pliku.
\begin{lstlisting}[language=C++, caption={Deklaracja wraz z inicjalizacją kontenera QMap<quint8, int>}]
QMap<quint8, int> IIMap;
for (quint8 i = 0; i < 255; i++) {
	IIMap.insert(i, 0);
}
\end{lstlisting}

W przypadku zdarzenia, że w pliku występują wszystkie możliwe kombinacje bajta, zamiast jedno bajtowego ,,znaku'' zostaje wprowadzony znak dwu bajtowy:
\begin{lstlisting}[language=C++, caption={Deklaracja dwu bajtowej zmiennej znakowej}]
QPair<quint8, quint8> SIGN16;
\end{lstlisting}

\subsubsection{Kodowanie}
\begin{figure}[H]
\centering
\begin{tikzpicture}[node distance = 2cm,auto]
    % Place nodes
    \node [block] (start) {Dopisz do tablicy pliku wyjściowego rozmiar ,,znaku'' oraz bajt/y ,,znak/u''};
    \node [block, below of=start] (jeden) {Wczytaj kolejny bajt};
    \node [decision, below of=jeden] (decyzja) {Czy bajt różni się od poprzedniego?};
    \node [block, right of=decyzja, node distance=5cm] (dwa) {Dopisz Bajt\\ do Listy Elementów};
    \node [block, left of=decyzja, node distance=5cm] (trzy) {Zinkrementuj liczbę\\ wystąpień ostatniego\\ elementu listy};
    \node [block, below of=decyzja, node distance=3.5cm] (cztery) {Zapisz Listę Elementów do Tablicy bajtów wyjściowych};
    \node [block, below of=cztery] (piec) {Zapisz tablicę do pliku};
	\node [block, below of=piec] (szesc) {Koniec.};    
    
    % Draw edges
    \path [line] (start) -- (jeden);
    \path [line] (jeden) -- (decyzja);
    \path [line] (decyzja) -- node [near start] {Tak} (dwa);
    \path [line] (decyzja) -- node [near start] {Nie} (trzy);
    \path [line] (decyzja) -- node [near start] {Koniec tablicy} (cztery);
    \path [line] (dwa) |- (jeden);
    \path [line] (trzy) |- (jeden);
    \path [line] (cztery) -- (piec);
    \path [line] (piec) -- (szesc);        
\end{tikzpicture}
\caption{Schemat blokowy kodowania pliku.}
\end{figure}

Do sprawnego wczytania, zliczenia i zakodowania poszczególnych bajtów stworzono kontener \lstinline;QList; struktury \lstinline;Element;. Struktura \lstinline;Element; posiada dwa pola: \lstinline;quint8 item; $\rightarrow$ oznaczające dany bajt oraz \lstinline;quint32 value; $\rightarrow$ oznaczające ilość wystąpień (Założono, że dany bajt nie powtórzy się więcej jak $4 294 967 295$ razy).
\begin{lstlisting}[language=C++, caption={Główne struktury danych kodowania}]
struct Element {
	quint8 item;
	quint32 value;
};
QList<Element> Elements;
quint8 CurrentByte;
\end{lstlisting}

Zapisanie danych do pliku odbywa się za pośrednictwem tablicy bajtów \lstinline;QByteArray;. Analiza zapisanych znaków odbywa się poprzez przejście przez wcześniej wspomnianą listę: \lstinline;QList<Element>; i odpowiednią interpretację  wartości wystąpień danego znaku. 

Jeżeli wartość powtórzenia danego znaku nie jest większa niż minimalna długość jego zastąpienia, który ma postać ,,znak'' inicjujący, bajt powtarzany, ilość powtórzeń, to wpisywana jest ,,pierwotna postać''.

W przypadku, gdy wartość ilości powtórzeń bajtu jest większa niż maksymalna wartość jaką może osiągnąć bajt - 255 - to postać zastąpienia przybiera postać: czterech bajtów ,,znaku'', bajt powtarzany i cztery bajty licznika.
\begin{lstlisting}[language=C++, caption={Zapisanie i zakodowania danych skompresowanych do pliku}]
if (e.value < 256) {
	OutByteArray.append(SIGN);
	OutByteArray.append(e.item);
	OutByteArray.append(e.value);
} else {
	OutByteArray.append(SIGN);
    OutByteArray.append(SIGN);
    OutByteArray.append(SIGN);
    OutByteArray.append(SIGN);
    OutByteArray.append(e.item);
    QByteArray TempArray;
    TempArray = RLE::IntToHex(e.value);
    OutByteArray.append(TempArray);
}
\end{lstlisting}

\subsection{Dekodowanie}
\begin{figure}[H]
\centering
\begin{tikzpicture}[node distance = 2cm,auto]
    % Place nodes
    \node [block] (start) {Wczytaj rozmiar ,,znaku''};
    \node [block, below of=start] (zero) {Wczytaj ,,znak''};
    \node [block, below of=zero] (jeden) {Wczytaj bajt};
    \node [decision, below of=jeden] (decyzja) {Czy bajt jest rónwy ,,znakowi''?};
    \node [block, right of=decyzja, node distance=5cm] (dwa) {Dopisz Bajt do\\ tablicy bajtów wyjściowych};
    \node [block, left of=decyzja, node distance=5cm] (trzy) {Wczytaj kolejny bajt\\ -powtarzany bajt-\\ i następny bajt -licznik\\ powtórzenia. Dopisz do\\ tablicy bajtów wyjściowych\\ powtarzany bajt w ilości\\ licznika powtórzenia.};
    \node [block, below of=decyzja, node distance=3cm] (cztery) {Zapisz tablicę do pliku};
    \node [block, below of=cztery] (piec) {Koniec.};
    
    % Draw edges
    \path [line] (start) -- (zero);
    \path [line] (zero) -- (jeden);
    \path [line] (jeden) -- (decyzja);
    \path [line] (decyzja) -- node [near start] {Nie} (dwa);
    \path [line] (decyzja) -- node [near start] {Tak} (trzy);
    \path [line] (decyzja) -- node [near start] {Koniec tablicy} (cztery);
    \path [line] (dwa) |- (jeden);
    \path [line] (trzy) |- (jeden);
    \path [line] (cztery) -- (piec);      
\end{tikzpicture}
\caption{Schemat blokowy dekodowania pliku.}
\end{figure}
Dekodowanie odbywa się z pomocą analogicznych struktur / zasad / metod co zostały użyte podczas kodowania.\\[2cm]
W celu ułatwienia kontroli nad plikami po kodowaniu mają one dodawany przyrostek \lstinline;.rlemama;, a podczas dekodowania dodawany przed znacznikiem formatu pliku przyrostek \lstinline;_2;
\newpage
\section{Użytkowanie programu}
Ikony umieszczone w programie zostały pobrane z strony: \url{http://www.flaticon.com/}\cite{ikony}.
\renewcommand{\figurename}{Zrzut ekranu} %Zmiana figure na Zrzut ekranu
\begin{figure}[H]
	\setcounter{figure}{0} %Resetownie licznika
	\centering
	\includegraphics[scale=.7]{start}
	\caption{Wygląd po ,,starcie'' programu.}
\end{figure}
\begin{figure}[H]
	\centering
	\includegraphics[scale=.7]{info}
	\caption{Włączenie informacji o autorach projektu.}
\end{figure}
\begin{figure}[H]
	\centering
	\includegraphics[scale=.7]{help}
	\caption{Włączenie okna pomocy.}
\end{figure}
\begin{figure}[H]
	\centering
	\includegraphics[scale=.7]{take}
	\caption{Przestawienie okna dialogowego wyboru plików, które program może skompresować.}
\end{figure}
\begin{figure}[H]
	\centering
	\includegraphics[scale=.7]{com}
	\caption{Przedstawienie plików gotowych do kompresji.}
\end{figure}
\begin{figure}[H]
	\centering
	\includegraphics[scale=.7]{take2}
	\caption{Przestawienie okna dialogowego wyboru plików, które program może zdekompresować.}
\end{figure}
\begin{figure}[H]
	\centering
	\includegraphics[scale=.7]{decom}
	\caption{Przedstawienie plików gotowych do dekompresji.}
\end{figure}
\begin{figure}[H]
	\centering
	\includegraphics[scale=.7]{all}
	\caption{Przedstawienie plików z dekompresowanych i nie skompresowanych}
\end{figure}

\begin{figure}[H]
	\centering
	\begin{minipage}{0.45\textwidth}
		\centering
		\includegraphics[scale=.4]{error1}
	\end{minipage}\hfill
	\begin{minipage}{0.45\textwidth}
		\centering
		\includegraphics[scale=.4]{error2}
	\end{minipage}
	\caption{Przedstawienie komunikatów błędów, gdy nakażemy z dekompresować / skompresować pliki ,,przemieszane''.}
\end{figure}
\vspace{1cm}
Dodatkowo w programie zaimplementowano mechanizmy ,,podglądu'' przetwarzanych plików.
\begin{figure}[H]
	\centering
	\begin{minipage}{0.45\textwidth}
		\centering
		\includegraphics[scale=.4]{audioviewer_clean}
	\end{minipage}\hfill
	\begin{minipage}{0.45\textwidth}
		\centering
		\includegraphics[scale=.4]{audioviewer_com}
	\end{minipage}
	\caption{Okno umożliwiające przesłuchanie utworów muzycznych przed i po kompresji.}
\end{figure}
\begin{figure}[H]
	\centering
	\begin{subfigure}{0.45\textwidth}
		\centering
		\includegraphics[scale=.4]{imageviewer_nor}
		\caption{Przed kompresją, bez skali.}
	\end{subfigure}\hfill
	\begin{subfigure}{0.45\textwidth}
		\centering
		\includegraphics[scale=.4]{imageviewer_min}
		\caption{Przed kompresją, minimalna skala.}
	\end{subfigure}\\[.5cm]
	\begin{subfigure}{0.45\textwidth}
		\centering
		\includegraphics[scale=.4]{imageviewer_max}
		\caption{Przed kompresją, maksymalna skala.}
	\end{subfigure}\hfill
	\begin{subfigure}{0.45\textwidth}
		\centering
		\includegraphics[scale=.4]{imageviewer_com}
		\caption{Po kompresji, bez skali.}
	\end{subfigure}
	\caption{Okno umożliwiające przesłuchanie podgląd plików graficznych.}	
\end{figure}

\section{Testy}
W celu przeprowadzenia testów wykonano kilka prostych obrazków formatu \lstinline;bmp; oraz pobrano z Internetu inne, większe i  bardziej skomplikowane obrazki. Dodatkowo do badań pobrano kilka plików dźwiękowych formatu \lstinline;wav; z strony \url{http://download.wavetlan.com/SVV/Media/HTTP/http-wav.htm}\cite{nuty}.
\begin{table}[H]
\centering
\begin{tabular}{l l l l l l}
\fbox{\includegraphics[scale=4]{test1}} & \fbox{\includegraphics[scale=4]{test2}} & \fbox{\includegraphics[scale=4]{test3}} & \fbox{\includegraphics[scale=4]{test4}}
& \fbox{\includegraphics[scale=4]{test5}} & \fbox{\includegraphics[scale=10]{test6}}
\end{tabular}
\caption{Przykładowe pliki graficzne (powiększone):}
\end{table}

\begin{minipage}{.48\textwidth}
\begin{lstlisting}[language=bash,caption={Przedstawienie plików przed kompresją.}]
NAZWA 		ROZMIAR [B]
audio.wav	3009870
audio2.wav	316002
audio3.wav	869028
audio4.wav	261262
highway.wav	8533723
test.bmp	126
test1.bmp	442
test2.bmp	442
test3.bmp	442
test4.bmp	442
test5.bmp	442
test6.bmp	170
test7.bmp	10922
test8.bmp	20138
test9.bmp	120122
test10.bmp	1163198
test11.bmp	44264
test12.bmp	309464
test13.bmp	16000138
test14.bmp	131554
test15.bmp	693122
\end{lstlisting}
\end{minipage}
\hfill
\begin{minipage}{.48\textwidth}
\begin{lstlisting}[language=bash,caption={Przedstawienie plików po kompresji.}]
NAZWA 				ROZMIAR [B]
audio.wav	3009870
audio2.wav	316002
audio3.wav	869028
audio4.wav	261262
highway.wav	8533723
test.bmp	126
test1.bmp	442
test2.bmp	442
test3.bmp	442
test4.bmp	442
test5.bmp	442
test6.bmp	170
test7.bmp	10922
test8.bmp	20138
test9.bmp	120122
test10.bmp	1163198
test11.bmp	44264
test12.bmp	309464
test13.bmp	16000138
test14.bmp	131554
test15.bmp	693122
\end{lstlisting}
\end{minipage}


\begin{minipage}{.48\textwidth}
\begin{lstlisting}[language=bash,caption={Przedstawienie plików przed kompresją.}]
NAZWA		ROZMIAR [B]
audio2.wav	316002
audio3.wav	869028
audio4.wav	261262
audio.wav	3009870
highway.wav	8533723
test10.bmp	1163198
test11.bmp	44264
test12.bmp	309464
test13.bmp	16000138
test14.bmp	131554
test15.bmp	693122
test1.bmp	442
test2.bmp	442
test3.bmp	442
test4.bmp	442
test5.bmp	442
test6.bmp	170
test7.bmp	10922
test8.bmp	20138
test9.bmp	500138
test.bmp	126
\end{lstlisting}
\end{minipage}
\hfill
\begin{minipage}{.48\textwidth}
\begin{lstlisting}[language=bash,caption={Przedstawienie plików po dekompresji.}]
NAZWA			ROZMIAR [B]
audio_2.wav		3009873
audio2_2.wav	316001
audio3_2.wav	869034
audio4_2.wav	261262
highway_2.wav	8533722
test_2.bmp		126
test1_2.bmp		445
test2_2.bmp		442
test3_2.bmp		442
test4_2.bmp		442
test5_2.bmp		442
test6_2.bmp		170
test7_2.bmp		10922
test8_2.bmp		20141
test9_2.bmp		120122
test10_2.bmp	1163198
test11_2.bmp	44270
test12_2.bmp	309464
test13_2.bmp	16000138
test14_2.bmp	131599
test15_2.bmp	693121
\end{lstlisting}
\end{minipage}

\subsection{Przedstawienie wyników:}
\begin{table}[H]
\begin{tabular}{r|l|l|l|l|l|}\hline
\textbf{PLIK}:	&	audio.wav	&	audio2.wav	&	audio3.wav	&	audio4.wav	&	highway.wav	\\\hline
\textbf{ROZMIAR}:	&	3009870	&	316002	&	869028	&	261262	&	8533723	\\\hline
\textbf{SKOMPRESOWANY}:	&	56	&	315928	&	856106	&	261239	&	8447839	\\\hline
\textbf{DEKOMPRESJA}:	&	3009873	&	316001	&	869034	&	261262	&	8533722	\\\hline
\textbf{KOMPRESJA \%}:	&	0,002\%	&	99,977\%	&	98,513\%	&	99,991\%	&	98,994\%	\\\hline
\end{tabular}
\end{table}

\begin{table}[H]
\begin{tabular}{r|l|l|l|l|l|}\hline
\textbf{PLIK}:	&	test.bmp	&	test1.bmp	&	test2.bmp	&	test3.bmp	&	test4.bmp	\\\hline
\textbf{ROZMIAR}:	&	126	&	442	&	442	&	442	&	442	\\\hline
\textbf{SKOMPRESOWANY}:	&	58	&	61	&	105	&	365	&	365	\\\hline
\textbf{DEKOMPRESJA}:	&	126	&	445	&	442	&	442	&	442	\\\hline
\textbf{KOMPRESJA \%}:	&	46,032\%	&	13,801\%	&	23,756\%	&	82,579\%	&	82,579\%	\\\hline
\end{tabular}
\end{table}


\begin{table}[H]
\begin{tabular}{r|l|l|l|l|l|}\hline
\textbf{PLIK}:	&	test5.bmp	&	test6.bmp	&	test7.bmp	&	test8.bmp	&	test9.bmp	\\\hline
\textbf{ROZMIAR}:	&	442	&	170	&	10922	&	20138	&	120122	\\\hline
\textbf{SKOMPRESOWANY}:	&	364	&	95	&	8216	&	15076	&	96180	\\\hline
\textbf{DEKOMPRESJA}:	&	442	&	170	&	10922	&	20141	&	120122	\\\hline
\textbf{KOMPRESJA \%}:	&	82,353\%	&	55,882\%	&	75,224\%	&	74,863\%	&	80,069\%	\\\hline
\end{tabular}
\end{table}


\begin{table}[H]
\begin{tabular}{r|l|l|l|l|l|l|} \hline
\textbf{PLIK}:	&	test10.bmp	&	test11.bmp	&	test12.bmp	&	test13.bmp	&	test14.bmp	&	test15.bmp	\\\hline
\textbf{ROZM.}:	&	1163198	&	44264	&	309464	&	16000138	&	131554	&	693122	\\\hline
\textbf{SKOM.}:	&	1163182	&	5689	&	85103	&	8440758	&	76746	&	345212	\\\hline
\textbf{DEKO.}:	&	1163198	&	44270	&	309464	&	16000138	&	131599	&	693121	\\\hline
\textbf{KOMP. \%}:	&	99,999\%	&	12,852\%	&	27,500\%	&	52,754\%	&	58,338\%	&	49,805\%	\\\hline
\end{tabular}
\end{table}


\begin{table}[H]
\centering
\begin{tabular}{c l l l l l l}1
PLIK: & test1.bmp & test2.bmp & test3.bmp & test4.bmp & test5.bmp & test6.bmp\\
PRZED: & \fbox{\includegraphics[scale=4]{test1}} & \fbox{\includegraphics[scale=4]{test2}} & \fbox{\includegraphics[scale=4]{test3}} & \fbox{\includegraphics[scale=4]{test4}} & \fbox{\includegraphics[scale=4]{test5}} & \fbox{\includegraphics[scale=10]{test6}} \\[.2cm] 
PO: & \fbox{\includegraphics[scale=4]{test1_2}} & \fbox{\includegraphics[scale=4]{test2_2}} & \fbox{\includegraphics[scale=4]{test3_2}} & \fbox{\includegraphics[scale=4]{test4_2}} & \fbox{\includegraphics[scale=4]{test5_2}} & \fbox{\includegraphics[scale=10]{test6_2}}
\end{tabular}
\caption{Porównanie plików graficznych przed i po kompresji (odpowiednio powiększone):}
\end{table}

\begin{table}[H]
\centering
\begin{tabular}{c l l l l l l}
PLIK: &test8.bmp&test9.bmp&test10.bmp&test11.bmp&test12.bmp&test13.bmp\\
PRZED & \fbox{\includegraphics[width=.1\textwidth]{test8}} & \fbox{\includegraphics[width=.1\textwidth]{test9}} & \fbox{\includegraphics[width=.1\textwidth]{test10}} & \fbox{\includegraphics[width=.1\textwidth]{test11}} & \fbox{\includegraphics[width=.1\textwidth]{test12}} & \fbox{\includegraphics[width=.1\textwidth]{test13}} \\[.2cm] 
PO & \fbox{\includegraphics[width=.1\textwidth]{test8_2}} & \fbox{\includegraphics[width=.1\textwidth]{test9_2}} & \fbox{\includegraphics[width=.1\textwidth]{test10_2}} & \fbox{\includegraphics[width=.1\textwidth]{test11_2}} & \fbox{\includegraphics[width=.1\textwidth]{test12_2}} & \fbox{\includegraphics[width=.1\textwidth]{test13_2}}
\end{tabular}
\caption{Porównanie niektórych plików graficznych przed i po kompresji (odpowiednio przeskalowane):}
\end{table}

Porównania bajtowe wykonano programem vbindiff\cite{diff}.
\begin{figure}[H]
	\centering
	\begin{minipage}{0.45\textwidth}
		\centering
		\includegraphics[scale=.3]{test1_beg}
	\end{minipage}\hfill
	\begin{minipage}{0.45\textwidth}
		\centering
		\includegraphics[scale=.3]{test1_end}
	\end{minipage}
	\caption{Porównanie różnic dwóch tablic bajtowych przed kompresją i po dekompresji.}
\end{figure}

\begin{figure}[H]
	\centering
	\begin{minipage}{0.45\textwidth}
		\centering
		\includegraphics[scale=.3]{audio2_beg}
	\end{minipage}\hfill
	\begin{minipage}{0.45\textwidth}
		\centering
		\includegraphics[scale=.3]{audio2_end}
	\end{minipage}
	\caption{Porównanie różnic dwóch tablic bajtowych przed kompresją i po dekompresji.}	
\end{figure}

\begin{figure}[H]
	\centering
	\begin{minipage}{0.45\textwidth}
		\centering
		\includegraphics[scale=.3]{highway_beg}
	\end{minipage}\hfill
	\begin{minipage}{0.45\textwidth}
		\centering
		\includegraphics[scale=.3]{highway_end}
	\end{minipage}
	\caption{Porównanie różnic dwóch tablic bajtowych przed kompresją i po dekompresji.}	
\end{figure}

\section{Wnioski}
Algorytm \textbf{Run-Length Encoding} doskonale się sprawdza do kompresji plików graficznych monochromatycznych (jedno kolorowych) - kompresuje pliki na poziomie $20-50 \%$. Przy bardziej złożonych - bardziej kolorowych - plikach graficznych kompresja spada do poziomu $80-99\%$. Pliki dźwiękowe, z racji swojej zawiłości również nie są podatne na kompresję tym algorytmem.

Algorytm jest prosty w implementacji i obsłudze oraz dla niektórych plików może dać ,,zachwycające'' rezultaty - kompresja na poziomie tysięcznych części pliku wejściowego. Jednakże nie gwarantuje on ,,oszczędności'' miejsca - kompresji.

W procesie kompresji / dekompresji błędy mogą powstać jedynie w skomplikowanych plikach o dużej złożoności pamięciowej / kolorystycznej / bajtowej, w innych nie powstały żadne znaczące wyjątki. ,,Przeinaczenia'' bajtowe powstałe wskutek ,,wypełnienia'' bajtami z dekompresowanych plików nie wpływają na obiekt końcowy.

\addcontentsline{toc}{section}{Literatura}  %dodanie znacznika "Literatura" do spisu treści
\begin{thebibliography}{1}
	\bibitem{latex} Kurs \LaTeX w $\pi^e$ minut \url{http://www.fuw.edu.pl/~kostecki/kurs_latexa.pdf}.
	\bibitem{program} Program Texmaker 4.5 \url{http://www.xm1math.net/texmaker/}.
	\bibitem{sharelatex} ShareLaTeX online LaTeX editor \url{https://www.sharelatex.com/}.	
	\bibitem{qt} Qt Creator – wieloplatformowe środowisko programistyczne \url{http://www.qt.io/}.
	\bibitem{gcc} GCC, the GNU Compiler Collection \url{https://gcc.gnu.org/}.
	\bibitem{git} Git rozproszony system kontroli wersji \url{http://git-scm.com/}.
	\bibitem{github} GitHub Web-based Git repository \url{https://github.com/}.
	\bibitem{golomb} Run-length encodings - S. W. Golomb (1966); IEEE Trans Info Theory 12(3):399 \url{http://urchin.earth.li/~twic/Golombs_Original_Paper/}.
	\bibitem{book} Variable-length codes for data compression / David Salomon, London : Springer, 2007.
	\bibitem{ikony} The largest database of free vector icons - flaticon \url{http://www.flaticon.com/}.
	\bibitem{nuty} Sample WAV files \url{http://download.wavetlan.com/SVV/Media/HTTP/http-wav.htm}.
	\bibitem{diff} vbindiff - hexadecimal file display and comparison \url{http://manpages.ubuntu.com/manpages/hardy/man1/vbindiff.1.html}.
\end{thebibliography}

	\blfootnote{Dokument wykonany w \LaTeX \cite{latex} \\ w programie Texmaker 4.5 \cite{program}}	
\end{document}
%\begin{lstlisting}[language=C++, caption={}]
%\end{lstlisting}
