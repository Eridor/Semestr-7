\documentclass[12pt,a4paper]{article}
\usepackage{polski}
\usepackage[utf8]{inputenc}
\usepackage[polish]{babel}
\usepackage{url}
\usepackage[pdftex]{graphicx}
\usepackage{sidecap}
\usepackage{verbatim}
\usepackage{array}
\usepackage{amsmath}
\usepackage{enumitem}
\usepackage{fancyhdr}
\usepackage{geometry}
\usepackage{float}
\usepackage{listings}
\usepackage{color}
\usepackage{wrapfig,booktabs}
\usepackage{float} %Do opcji H w figure i table

\newcommand\blfootnote[1]{%
  \begingroup
  \renewcommand\thefootnote{}\footnote{#1}%
  \addtocounter{footnote}{-1}%
  \endgroup
}
%remove section whitespace
%\usepackage{titlesec}
%\titlespacing\section{0pt}{12pt plus 4pt minus 2pt}{0pt plus 2pt minus 2pt}

%\makeatletter
%\newcommand*{\toccontents}{\@starttoc{toc}}
%\makeatother

\renewcommand\thesection{\arabic{section}}
\newcommand{\HRule}{\rule{\linewidth}{0.5mm}} %Do tytułu

\usepackage{etoolbox} %do bibliografi
%\patchcmd{\thebibliography}{\chapter*}{\section*}{}{} %żeby bibliografię nie wrzucało na nową stronę

%Do kodu C++
\usepackage{xcolor} % for setting colors

% set the default code style
\lstset{
    frame=tb, % draw a frame at the top and bottom of the code block
    tabsize=4, % tab space width
    showstringspaces=false, % don't mark spaces in strings
    numbers=left, % display line numbers on the left
    commentstyle=\color{green}, % comment color
    keywordstyle=\color{blue}, % keyword color
    stringstyle=\color{red} % string color
}
%End kod

\begin{document}
\renewcommand\tablename{Tabela} %zmienia Tablica na "Tabela"

\begin{titlepage}
\begin{center}

\textsc{\LARGE Politechnika Poznańska}\\[0.2cm]
\textsc{\Large Wydział Elektryczny \\ Informatyka}\\[0.6cm]

\includegraphics[scale=0.8]{logopoli}
\vspace{.5cm}

\textsc{\large Teoria Informacji i Kodowanie}\\[0.2cm]
\textsc{\normalsize Projekt semestralny}\\[.2cm]
\textsc{\large Projekt implementujący algorytm kompresji i dekompresji RLE (ang.: \textit{Run Length Encoding}).}\\[.5cm]


% Author and supervisor
\noindent
\begin{minipage}[t]{0.49\textwidth}
\begin{flushleft} \large
\emph{Autorzy:}\\
Michał \textsc{Majka} \\
Nr albumu: 666666 \\
Piotr \textsc{Parysek} \\
Nr albumu: 106100
\end{flushleft}
\end{minipage}%
\begin{minipage}[t]{0.49\textwidth}
\begin{flushright} \large
\emph{Prowadzący:} \\
dr inż. Ewa \textsc{Idzikowska}.
\end{flushright}
\end{minipage}

\bigskip
\vspace{\fill}

\begin{minipage}[b]{\textwidth}
    \centering
%    \onehalfspacing
    \large   
    {\large \today}\\
    {\large Rok akademicki 2015/2016}

%    \vspace{-20mm} 
\end{minipage}%

% Bottom of the page
%{\large \today}
%\global\let\newpagegood\newpage %żeby nie było page break po titilepage
%\global\let\newpage\relax %żeby nie było page break po titilepage

\end{center}
\end{titlepage}

\tableofcontents

\addcontentsline{toc}{section}{Literatura}  %dodanie znacznika "Literatura" do spisu treści
\begin{thebibliography}{3}
	\bibitem{latex} Kurs \LaTeX w $\pi^e$ minut \url{http://www.fuw.edu.pl/~kostecki/kurs_latexa.pdf}.
	\bibitem{program} Program Texmaker 4.4.1 \url{http://www.xm1math.net/texmaker/}.
	\bibitem{uml} Visual Paradigm Community Edition 12.2 \url{www.visual-paradigm.com}.
	\bibitem{ikony} Ikony \url{http://www.flaticon.com/}.

\end{thebibliography}
	\blfootnote{Dokument wykonany w \LaTeX \cite{latex} \\ w programie Texmaker\cite{program}}
\end{document}